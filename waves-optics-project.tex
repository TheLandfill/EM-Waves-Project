\documentclass[12pt]{article}

\usepackage{fancyhdr}
\usepackage{extramarks}
\usepackage{amsmath} %\usepackage{amsthm}
\usepackage{amsfonts}
\usepackage{tikz}
%\usepackage{gensymb}
%\usepackage[plain]{algorithm}
\usepackage{cancel}
%\usepackage{algpseudocode}
%\usepackage{mathtools}
\usepackage{empheq}
\usepackage{siunitx}
\usepackage{bm}
\usepackage{physics}

\usetikzlibrary{automata,positioning}
\usetikzlibrary{arrows.meta}

%
% Basic Document Settings
%

\topmargin=-0.45in
\evensidemargin=0in
\oddsidemargin=0in
\textwidth=6.5in
\textheight=9.0in
\headsep=0.25in

\linespread{1.1}

\pagestyle{fancy}
\lhead{\hmwkAuthorName}
\chead{\hmwkClass\ \hmwkTitle}
\rhead{\firstxmark}
\lfoot{\lastxmark}
\cfoot{\thepage}

\renewcommand\headrulewidth{0.4pt}
\renewcommand\footrulewidth{0.4pt}

\DeclareMathOperator{\arccosh}{arccosh}

\setlength\parindent{0pt}

%
% Create Problem Sections
%

\newcommand{\enterProblemHeader}[1]{
	\rhead{#1}
}

\newcommand{\exitProblemHeader}[1]{
	\pagebreak
}

%
% Homework Problem Environment
%
% This environment takes an optional argument. When given, it will adjust the
% problem counter. This is useful for when the problems given for your
% assignment aren't sequential. See the last 3 problems of this template for an
% example.
%
\newenvironment{homeworkProblem}[1]{
    \section{Problem #1}
    \enterProblemHeader{#1}
}{
	\pagebreak
}

\newenvironment{main_section}[1]{
	\section{#1}
	\enterProblemHeader{#1}
}{
	\pagebreak
}

%
% Homework Details
%   - Title
%   - Due date
%   - Class
%   - Section/Time
%   - Instructor
%   - Author
%

\newcommand{\hmwkTitle}{Project}
\newcommand{\hmwkDueDate}{April 29, 2020}
\newcommand{\hmwkClass}{Waves \& Optics}
\newcommand{\hmwkClassInstructor}{Prof. Peifen Zhu}
\newcommand{\hmwkAuthorName}{\textbf{Joseph Mellor}}

%
% Title Page
%

\title{
    \vspace{2in}
    \textmd{\textbf{\hmwkClass:\ \hmwkTitle}}\\
    \normalsize\vspace{0.1in}\small{Due\ on\ \hmwkDueDate}\\
    \vspace{0.1in}\large{\textit{\hmwkClassInstructor}}
    \vspace{3in}
}

\author{\hmwkAuthorName}
\date{}

%
% Various Helper Commands
%

% Useful for algorithms
\newcommand{\alg}[1]{\textsc{\bfseries \footnotesize #1}}

% For derivatives
\newcommand{\deriv}[2]{\frac{d#1}{d#2}}

% For partial derivatives
\newcommand{\pderiv}[2]{\frac{\partial #1}{\partial #2}}

% Integral dx
\newcommand{\dx}[1]{\mathrm{d}#1}

% Alias for the Solution section header
\newcommand{\solution}{\textbf{\large Solution}}

% Probability commands: Expectation, Variance, Covariance, Bias
\newcommand{\E}{\mathrm{E}}
\newcommand{\Var}{\mathrm{Var}}
\newcommand{\Cov}{\mathrm{Cov}}
\newcommand{\Bias}{\mathrm{Bias}}

\newcommand{\R}{\mathbb{R}}
\newcommand{\Z}{\mathbb{Z}}
\newcommand{\Q}{\mathbb{Q}}
\newcommand{\N}{\mathbb{N}}
\newcommand{\C}{\mathbb{C}}

\newcommand{\tensor}[1]{\overset{\leftrightarrow}{#1}}

\newcommand\ddfrac[2]{\frac{\displaystyle #1}{\displaystyle #2}}
\DeclarePairedDelimiter\ceil{\lceil}{\rceil}
\DeclarePairedDelimiter\floor{\lfloor}{\rfloor}

\begin{document}

\maketitle

\pagebreak

\begin{main_section}{Goal}
	In order to maximize the light extraction efficiency, we must maximize the
	total amount of light transmitted from the point source through the top of
	the LED. The light escaping through the sides will be negligible.

	\subsection{Model}

	We are going to model the LED as a point source in a medium of constant
	refractive index, with one side reflecting all light that hits it and the
	other side reflecting and transmitting the light based on the physical
	properties at the boundary. We can model reflections as having a virtual
	point source on the opposite side of the mirror with the materials also
	being mirrored, and we will use this model to make our calculations simpler.
	Lastly, we will model the entire LED as one medium since it has the same
	exact index of refraction throughout it. We will use the symbols $d$ to
	represent the entire thickness of the LED and $d_0$ to represent the
	distance from the Air-LED Boundary.

	\definecolor{crimson}{HTML}{DC143C}
	\definecolor{cool-crimson}{HTML}{B91161}
	\definecolor{mid-red}{HTML}{F31AA0}
	\definecolor{cool-pink}{HTML}{E30BE1}
	\definecolor{cool-violet}{HTML}{A109D0}
	\definecolor{air}{HTML}{D5D4FF}
	\definecolor{light-brown}{HTML}{E0E284}
	\definecolor{light-green}{HTML}{F7F79C}
	\definecolor{silver}{HTML}{C0C0C0}

	\begin{figure}[h]
	\centering
	\begin{tikzpicture}[scale=1.3]
		\fill[air] (-4, 1) rectangle (4, 3);
		\fill[air] (-4, 3) -- (-3, 4) -- (5, 4) -- (4, 3);
		\fill[air] (5, 4) -- (4, 3) -- (4, 1) -- (5, 2);
		\fill[light-green] (-4, -3) rectangle (4, 1);
		\fill[light-green] (-4, 1) -- (4, 1) -- (5, 2) -- (-3, 2);
		\fill[light-green] (4, -3) -- (5, -2) -- (5, 2) -- (4, 1);
		\fill[silver] (-4, -1) -- (-3, 0) -- (5, 0) -- (4, -1);
		\foreach \a in {-2, 0, 2}
			\draw[dashed] (-3, \a) -- (5, \a);
		\foreach \d in {0.2, 0.5, 0.8}
			\foreach \b/\c in {-2/cool-violet, 0/cool-crimson}
				\foreach \a in {-3,-1.5,...,3}
					\draw[\c,line width=0.4mm,dotted,-{Latex[length=3mm]}] (0 + 0.5, \b + 0.5) -- (\a + \d, 1
					+ \d);
		\draw (4, 1) -- (-4, 1) node [anchor=east, inner sep = 20pt] {Air-LED Boundary};
		\draw[cool-crimson,fill=cool-crimson] (0.5, 0.5) circle (0.12);
		\node[draw,fill=white] at (0.5, 0)
		{{\boldmath$\color{cool-crimson}{P_0}$}};
		\draw (4, -1) -- (-4, -1) node [anchor=east, inner sep = 20pt] {Silver Reflector};
		\draw[cool-violet,fill=cool-violet] (0.5, -1.5) circle (0.12);
		\node[draw,fill=white] at (0.5, -2)
		{{\boldmath$\color{cool-violet}{P_1}$}};
		\foreach \a in {-3,-1,1}
			\foreach \b/\c in {-4/dashed, 4/}
				\draw[\c] (\b, \a) -- (\b + 1, \a + 1);
		\draw[dashed] (-3, -2) -- (-3, 2);
		\draw (-4, -3) rectangle (4, 1);
		\draw (5, -2) -- (5, 2);
		\draw[|<->|,line width=0.3mm] (-4.2, -1) -- (-4.2, 0) node[midway, left]
		{$d - d_0$};
		\draw[|<->|,line width=0.3mm] (-4.2, 0) -- (-4.2, 1) node[midway, left]
		{$d_0$};
		\draw[|<->|,line width=0.3mm] (-4.2, -2) -- (-4.2, -3) node[midway,
		left] {$d_0$};
		\draw[|<->|,line width=0.3mm] (-4.2, -2) -- (-4.2, -1) node[midway,
		left] {$d - d_0$};
	\end{tikzpicture}
	\caption{Our Model}
	\end{figure}

	The light at $P_0$ and $P_1$ should produce the exact same amount of light.

	\subsection{A Change in Approach}

	Originally, I was going to consider a large number of virtual point sources
	as specified in the subsection \textbf{Higher Order Model}, but, as I'll
	later show, a majority of th elight within the critical angle will
	eventually pass through the surface, so the higher order models won't really
	be necessary for this problem. I actually did a bunch of calculations and
	wrote some integration software, but my math was wrong, so I dropped those
	results.

	\subsection{Higher Order Model}

	There will also be virtual point sources below $P_1$ because the light
	reflected at the Air-LED Boundary will then bounce off the silver reflector,
	some of which will bounce off the Air-LED Boundary, which will then bounce
	off the silver and so on. It will be an infinite sum of virtual point
	sources with less light as they get farther from the Air-LED Boundary. The
	distance away from the Air-LED Boundary for the $n^\text{th}$ will be
	\[
		d_n =
		\begin{cases}
		d_0 + nd, & n\quad\text{even}\\
		d - d_0 + nd, & n\quad\text{odd}
		\end{cases}
	\]
	Furthermore, the intensity of light for the $n^\text{th}$ point source will
	be
	\begin{align*}
		P_n(\theta_I, r) = P_0 (R(\theta_I))^{\lfloor n/2 \rfloor}
	\end{align*}
	where $R$ is the $i^\text{th}$ reflection coefficient for a given
	$\theta_I$ and $P_0$ is the power of the real power source.
\end{main_section}

\begin{main_section}{Initial Efficiency}
	We'll need Eqs. (9.110), (9.106), (9.116), and (9.200) to get $\alpha$,
	$\beta$, $R$ (reflection coefficient), $T$ (transmission coefficient), and
	$\theta_C$ (the critical angle).
	\begin{align*}
		\alpha &= \frac { \sqrt{1 - \beta^{-2} \sin^2 \theta_I} } {\cos \theta_I}
		\tag{9.110}\\
		\beta &= \frac {\mu_1 n_2} {\mu_2 n_1} \approx 0.4 \tag{9.106}\\
		R &= \left( \frac { \alpha - \beta } { \alpha + \beta } \right)^2
		\tag{9.115}\\
		T &= \alpha \beta \left( \frac 2 {\alpha + \beta} \right)^2\tag{9.116}\\
		\theta_C &= \arcsin \beta \tag{9.200} \approx 23.578^\circ
	\end{align*}
	where we made the simplification $\mu_1 = \mu_2$.

	\subsection{Using the Critical Angle}

	Any light outside of the critcal angle will be internally reflected, so we
	can ignore it, meaning our integration region will be smaller for lower
	order point sources (and possibly circular). Since $d_n$ increases
	approximately linearly, we can find an upper bound on $n$ where our
	integration region is a circle by considering $n d \tan \theta_C =
	x_\text{max}$. We can also use $d < 10 \si{\mu m}$ to get a hard upper
	bound.
	\begin{align*}
		n (10 \si{\mu m}) \tan(23.578^\circ) &< 300 \si{\mu m}\\
		n < \frac {30} {\tan(23.578^\circ)}\\
		n < 68.7\\
		n < 69 % Nice
	\end{align*}
	So for 68 internal reflections, all the light that can pass through will be
	within a circle of radius $d_n \tan \theta_C$.

	\subsection{Max Efficiency from Critical Angle}

	The critical angle also puts a maximum limit on the fraction of power that
	can be transmitted out of the LED. Since the point source is isotropic, the
	power is spread evenly over the surface of a sphere, $4 \pi R^2$. The only
	power that can escape, however, is within the critical angle, meaning only
	\begin{align*}
		&\frac 1 {4 \pi R^2} \left( 2 \int_{\varphi = 0}^{\varphi = 2\pi} \int_{r
		= 0}^{r = \infty} \int_{\theta = 0}^{\theta = \theta_C} r^2 \delta(r -
		R) \sin \theta d\theta dr d\varphi \right)\\
		&\frac 1 {2 \pi R^2} \left( \int_0^{2\pi} d\varphi \right) \left(
		\int_0^\infty \delta(r - R)r^2 dr \right) \left( \int_0^{\theta_C} \sin
		\theta d\theta \right)\\
		&= \frac 1 {2 \pi R^2} (2 \pi) (R^2) \left( \cos \theta
		\right|_{\theta_C}^0\\
		&= 1 - \cos \theta_C\\
		&\approx 0.08348 \approx 8.348\%
	\end{align*}

	8.348 \% of the power provided to the LED is available to be emitted from
	the LED. The rest of the power is internally reflected and heats up the LED.
	Furthermore, since there are at least 68 internal reflections (half of which
	are in contact with the surface), most of the light produced within the
	critical angle will be emitted.

\end{main_section}

\begin{main_section}{Increasing Extraction Efficiency}
	The best way to improve the LED's efficiency is to either increase the
	critical angle itself or to increase the amount of the LED-Air Boundary that
	is in contact with the light at an angle of less than the critical angle.
	Increasing the critical angle requires changing the material of the LED to a
	lower refractive index (which isn't likely). Alternatively, by changing the
	structure of the LED to a hemisphere with the bottom two layers being as
	thin as possible will mean that all light will be perpendicular to the
	surface of the LED and therefore be at the lowest possible critical angle.
	Doing so, however, would mean the LED would be around 30 times thicker.
\end{main_section}

\begin{main_section}{Thickness of Layers}
	If $d_2$ were $\lambda_\text{medium}/4$, the wave coming from the bottom of
	the point source would be $180^\circ$ out of phase with the wave coming from
	the top of the point source, meaning the electric field immediately above
	the point source would be near zero and therefore the electromagnetic energy
	in that region would also be near zero. By the conservation of energy, the
	electromagnetic energy outside of the LED would have to increase, meaning an
	increase in the power of the LED. I would estimate a $d_2$ of around
	\[
		\frac 1 4 \frac \lambda n = \frac 14 \frac {500 \si{nm}} {2.5} = 50
		\si{nm}
	\]
	Now, it could be the case that we want to minimize the reflection at another
	angle near zero and not necessarily zero. Doing so would require a thinner
	p-GAN layer in the LED to compensate for the horizontal distance traveled,
	but it would be around 50nm.
\end{main_section}

\newcommand{\nled}{n_\text{LED}}
\newcommand{\nmat}{n_\text{Material}}
\newcommand{\nair}{n_\text{Air}}
\newcommand{\rledmat}{R_\text{LED-Material}}
\newcommand{\tledmat}{T_\text{LED-Material}}
\newcommand{\rmatair}{R_\text{Material-Air}}

\begin{main_section}{Adding Materials on LED}
	While adding flat materials to the top of an LED won't change the critical
	angle since
	\begin{align*}
		&\nled \sin \theta_{C1} = \nmat \sin \theta_{C2} = 1\\
		&\nled \sin \theta_{C1} = 1
	\end{align*}
	which was the original condition for the critical angle, we can combine the
	idea of using a hemisphere with a medium with a lower index of refraction to
	flatten out the hemisphere, which will allow us to arbitrarily increase the
	crtical angle of the Material-Air Boundary. I actually can't come up with
	the equation for the shape of this lens, but it would have a reflection
	coefficient of
	\begin{align*}
		R_\text{eff} &= \rledmat + \tledmat * \rmatair\\
		&= \left( \frac {\nled - \nmat} {\nled + \nmat} \right)^2 + \left( \frac
		{4 \nled \nmat} {\left( \nled + \nmat \right)^2} \right) \left( \frac
		{\nmat - \nair} {\nmat + \nair} \right)^2
	\end{align*}
	To minimize the reflectio coefficient, I can take the derivative of the
	above expression, set it equal to zero, and solve for $\nmat$ in terms of
	$\nled$ and $\nair$.
	\[
		\pderiv { R_\text{eff} } { \nmat } = \frac {32 \nair \nled \nmat
		(\nmat^2 - \nair \nled)} { (\nled + \nmat)^3 (\nmat + \nair)^3 }
	\]
	which equals zero when $\nmat = 0$ or when $\nmat = \sqrt{ \nair \nled }
	\approx 1.5811$, only the second of which is possible. The minimum possible
	reflection coefficient is therefore
	\begin{align*}
		&R_\text{eff} \approx 0.0988\\
		&\implies T_\text{eff} \approx 0.9111
	\end{align*}
	With just the air and the LED, the reflection coefficient when the light is
	normal to the plane is
	\begin{align*}
		R &\approx 0.1837\\
		T &\approx 0.8163
	\end{align*}
	If we stacked more layers of more materials on the LED, we could get even
	higher transmission coefficients.

	\subsection{Internal Destructive Interference}
	If we couldn't use a lens, we could also use multiple layers of varying
	thicknesses to force waves and their reflections to be out of phase and
	cancel out inside the layers, which would mean the energy would have to
	escape from the top.
\end{main_section}
\end{document}
